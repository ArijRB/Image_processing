\documentclass[a4paper]{article}

%% Language and font encodings
\usepackage[french]{babel}
\usepackage[utf8]{inputenc}
\usepackage[T1]{fontenc}

\usepackage{float}

\setlength{\parindent}{1em}
%\setlength{\parskip}{1ex plus 0.5ex minus 0.2ex}
\newcommand{\hsp}{\hspace{20pt}}
\newcommand{\HRule}{\rule{\linewidth}{0.5mm}}

\usepackage{algorithm}
\usepackage[noend]{algpseudocode}
\algnewcommand{\algorithmicand}{\textbf{ and }}
\algnewcommand{\algorithmicor}{\textbf{ or }}
\algnewcommand{\OR}{\algorithmicor}
\algnewcommand{\AND}{\algorithmicand}
\algnewcommand\algorithmicforeach{\textbf{for each}}
\algdef{S}[FOR]{ForEach}[1]{\algorithmicforeach\ #1\ \algorithmicdo}
\newcommand{\myfrac}[2]{\frac{\displaystyle {#1}}{\displaystyle {#2}}}

%% Sets page size and margins
\usepackage[a4paper,top=3cm,bottom=2cm,left=3cm,right=3cm,marginparwidth=1.75cm]{geometry}

%% Useful packages
\usepackage{amsmath}
\usepackage{amssymb}
\usepackage{graphicx}
\usepackage{subcaption}
\usepackage[colorinlistoftodos]{todonotes}
\usepackage[colorlinks=true, allcolors=blue]{hyperref}
\usepackage{graphicx}

\usepackage{enumitem}
\setitemize{label=\textbullet, font=\small}

%% equations
\usepackage{amsthm}
\usepackage[retainorgcmds]{IEEEtrantools}

%% theorem and proposition
\newtheorem{prop}{Proposition}
\newtheorem*{prop*}{Proposition}
\newtheorem{thm}{Théorème}

\newenvironment{myproof}[1][\proofname]{\proof[#1]\mbox{}\\*}{\endproof}

%% references shortcuts (Arthur) 
\usepackage{suffix}
\renewcommand{\eqref}[1]{équation~\ref{#1}}
\newcommand{\algoref}[1]{algorithme~\ref{#1}}
\newcommand{\figref}[1]{figure~\ref{#1}}
\newcommand{\tabref}[1]{tableau~\ref{#1}}
\newcommand{\secref}[1]{section~\ref{#1}}
\newcommand{\probref}[1]{problème~\ref{#1}}
\newcommand{\propref}[1]{proposition~\ref{#1}}
\newcommand{\theoremref}[1]{théorème~\ref{#1}}
\newcommand{\chapref}[1]{chapitre~\ref{#1}}
\WithSuffix\newcommand\algoref*[1]{algorithme~\ref{#1} p.~\pageref{#1}}
\WithSuffix\newcommand\figref*[1]{figure~\ref{#1} p.~\pageref{#1}}
\WithSuffix\newcommand\eqref*[1]{équation~\ref{#1} p.~\pageref{#1}}
\WithSuffix\newcommand\tabref*[1]{tableau~\ref{#1} p.~\pageref{#1}}
\WithSuffix\newcommand\secref*[1]{section~\ref{#1} p.~\pageref{#1}}
\WithSuffix\newcommand\probref*[1]{problème~\ref{#1} p.~\pageref{#1}}
\WithSuffix\newcommand\propref*[1]{proposition~\ref{#1} p.~\pageref{#1}}
\WithSuffix\newcommand\chapref*[1]{chapitre~\ref{#1} p.~\pageref{#1}}

\usepackage[backend=biber,uniquename=init,giveninits=true,
             %% "et al" pour > deux auteurs, & pour exactement 2
             uniquelist=false,maxcitenames=2,mincitenames=1,maxbibnames=99,
             isbn=false,url=false,doi=false,bibstyle=numeric
]{biblatex}
\addbibresource{references.bib}

\begin{document}

\begin{titlepage}
  \begin{center}

      \makebox[0.5\textwidth][r]{%
        \includegraphics[width=0.33\textwidth]{images/sorbonne.png}%
    }%

      \vspace{4cm}
    % Title
    \HRule \\[0.4cm]
    { \huge \bfseries BIMA\\[0.4cm] }

      \textsc{\LARGE Mini rapport TP4}\\[0.4cm]

    \HRule \\[0.8cm]

    % Author and supervisor
    \begin{minipage}{0.4\textwidth}
      \begin{flushleft} \large
        Kim-Anh Laura \textsc{Nguyen}\\
        \large
        Arij \textsc{Riabi}\\
        M1 DAC\\
        Promo 2018-2019 \\
      \end{flushleft}
    \end{minipage}
    \begin{minipage}{0.5\textwidth}
      \begin{flushright} \large
        \emph{Enseignant :} Dominique \textsc{Béréziat}\\
      \end{flushright}
    \end{minipage}

      \vspace{2cm}

  \end{center}
  %\end{sffamily}
\end{titlepage}
%\maketitle

\newpage

\section*{Exercice 1 - Filtrage fréquentiel}

La \figref{fig:filtrage-mandrill1} (resp. \ref{fig:filtrage-lena1}) montrent les
étapes du filtrage passe-bas de l'image \ref{subfig:mandrill1} (resp.
\ref{subfig:lena1}).

\begin{figure}[H]
    \centering
    \begin{subfigure}[c]{0.46\textwidth}
        \centering
        \includegraphics[width=\textwidth]{images/mandrill.png}
        \caption{Image originale} 
    \label{subfig:mandrill1}
    \end{subfigure}
    \begin{subfigure}[c]{0.46\textwidth}
        \centering
        \includegraphics[width=\textwidth]{images/mandrill_FT1.png}
        \caption{Module centré de la transformée de Fourier de l'image}
    \label{subfig:mandrill-FT1}
    \end{subfigure}

    \begin{subfigure}[c]{0.46\textwidth}
        \centering
        \includegraphics[width=\textwidth]{images/mandrill_FT_filtre1.png}
        \caption{Module centré de la transformée de Fourier filtrée} 
        \label{subfig:mandrill-FT-filtre1}
    \end{subfigure}
    \begin{subfigure}[c]{0.46\textwidth}
        \centering
        \includegraphics[width=\textwidth]{images/mandrill_filtre1.png}
        \caption{Image filtrée}
    \label{subfig:mandrill-filtre1}
    \end{subfigure}

    \caption{Étapes du filtrage passe-bas de l'image \texttt{mandrill.png} avec $f_c = 30$}
    \label{fig:filtrage-mandrill1}
\end{figure}

\begin{figure}[H]
    \centering
    \begin{subfigure}[c]{0.46\textwidth}
        \centering
        \includegraphics[width=\textwidth]{images/lena.png}
        \caption{Image originale} 
    \label{subfig:lena1}
    \end{subfigure}
    \begin{subfigure}[c]{0.46\textwidth}
        \centering
        \includegraphics[width=\textwidth]{images/lena_FT1.png}
        \caption{Module centré de la transformée de Fourier de l'image}
    \label{subfig:lena-FT1}
    \end{subfigure}

    \begin{subfigure}[c]{0.46\textwidth}
        \centering
        \includegraphics[width=\textwidth]{images/lena_FT_filtre1.png}
        \caption{Module centré de la transformée de Fourier filtrée} 
    \label{subfig:lena-FT-filtre1}
    \end{subfigure}
    \begin{subfigure}[c]{0.46\textwidth}
        \centering
        \includegraphics[width=\textwidth]{images/lena_filtre1.png}
        \caption{Image filtrée}
    \label{subfig:lena-filtre1}
    \end{subfigure}

    \caption{Étapes du filtrage passe-bas de l'image \texttt{lena.png} de
    fréquence de coupure $f_c = 30$}
    \label{fig:filtrage-lena1}
\end{figure}

\begin{figure}[H]
    \centering
    \begin{subfigure}[c]{0.46\textwidth}
        \centering
        \includegraphics[width=\textwidth]{images/lena_filtre2.png}
        \caption{Image \texttt{lena.jpg} filtrée} 
    \label{subfig:lena-filtre-2}
    \end{subfigure}
    \begin{subfigure}[c]{0.46\textwidth}
        \centering
        \includegraphics[width=\textwidth]{images/mandrill_filtre2.png}
        \caption{Image \texttt{mandrill.png} filtrée} 
    \label{subfig:mandrill-filtre2}
    \end{subfigure}
    \caption{Images filtrées avec un filtre passe-bas de fréquence de coupure
    $f_c$ = 10}
    \label{fig:filtrage-lena1}
\end{figure}


Lorsque l'on diminue la fréquence de coupure $f_c$, les changements brusques
d'intensité sont atténués et l'image reconstruite présente plus de flou sur le
contour. \\

Ce type de filtrage fréquentiel est utilisé :
\begin{itemize}
    \item comme filtre anti-repliement dans la numérisation des signaux
    \item pour filtrer le bruit.
\end{itemize}

\section*{Exercice 2 - Filtrage linéaire}

Pour un filtre de taille $d$, on ajoute de part et d'autre de l'image $d-1$
lignes et $d-1$ colonnes de zéros.

La fonction \texttt{convolution} est testée sur l'image \texttt{barbara.png}
avec les filtres moyenneurs $3 \times 3$, $5 \times 5$, $7 \times 7$. Les images
résultantes sont présentées dans la \figref{fig:filtre-moy}.

\begin{figure}[H]
    \centering
    \begin{subfigure}[c]{0.3\textwidth}
        \centering
        \includegraphics[width=\textwidth]{images/filtre_moy_3.png}
        \caption{Application du filtre moyenneur $3 \times 3$} 
        \label{subfig:filtre-moy-3}
    \end{subfigure}
    \begin{subfigure}[c]{0.3\textwidth}
        \centering
        \includegraphics[width=\textwidth]{images/filtre_moy_5.png}
        \caption{Application du filtre moyenneur $5 \times 5$} 
        \label{subfig:filtre-moy-5}
    \end{subfigure}
    \begin{subfigure}[c]{0.3\textwidth}
        \centering
        \includegraphics[width=\textwidth]{images/filtre_moy_7.png}
        \caption{Application du filtre moyenneur $7 \times 7$} 
        \label{subfig:filtre-moy-7}
    \end{subfigure}
    \caption{Application de filtres moyenneurs sur l'image \texttt{barbara.png}}
    \label{fig:filtre-moy}
\end{figure}

On constate que plus la taille du filtre est grande, plus le lissage est important et
plus l'image filtrée perd en détails par rapport à l'image originale.\\

Les fonctions de transfert correspondants à ces filtres sont contenues dans la
\figref{fig:transfert-moy}.

\begin{figure}[H]
    \centering
    \begin{subfigure}[c]{0.3\textwidth}
        \centering
        \includegraphics[width=\textwidth]{images/transfert_moy_7.png}
        \caption{Fonction de transfert du filtre moyenneur $3 \times 3$} 
        \label{subfig:transfert-moy-3}
    \end{subfigure}
    \begin{subfigure}[c]{0.3\textwidth}
        \centering
        \includegraphics[width=\textwidth]{images/transfert_moy_5.png}
        \caption{Fonction de transfert du filtre moyenneur $5 \times 5$} 
        \label{subfig:transfert-moy-5}
    \end{subfigure}
    \begin{subfigure}[c]{0.3\textwidth}
        \centering
        \includegraphics[width=\textwidth]{images/transfert_moy_7.png}
        \caption{Fonction de transfert du filtre moyenneur $7 \times 7$} 
        \label{subfig:transfert-moy-7}
    \end{subfigure}
    \caption{Fonction de transfert des filtres moyenneurs}
    \label{fig:transfert-moy}
\end{figure}

La fonction de transfert d'un filtre moyenneur de taille $d$ est :

$$ H(f) = d \cdot sinc(\pi fa)$$

Le filtre moyenneur est un filtre passe-bas qui a l'inconvénient d'être très peu
sélectif : si l'on augmente $d$, on réduit encore plus les hautes fréquences
mais on altère aussi les basses fréquences. Il n'est donc pas idéal.

\section*{Exercice 3 - Filtrage anti-aliasing}

Le sous-échantillonnage introduit nécessairement des effets d'aliasing. Afin de
limiter ce problème, on effectue donc un filtrage passe-bas avant de
sous-échantillonner. La \figref{fig:filtrage-anti-aliasing} contient l'image
résultante après filtrage et sous-échantillonnage de l'image
\texttt{barbara.png} (\ref{subfig:barbara-avec-anti-aliasing}), et celle générée
par un sous-échantillonnage brutal (\ref{subfig:barbara-sans-anti-aliasing}).

\begin{figure}[H]
    \centering
    \begin{subfigure}[c]{0.46\textwidth}
        \centering
        \includegraphics[width=\textwidth]{images/barbara_sous_echantillonage_avec_anti_aliassing.png}
        \caption{Sous-échantillonnage avec anti-aliasing} 
        \label{subfig:barbara-avec-anti-aliasing}
    \end{subfigure}
    \begin{subfigure}[c]{0.46\textwidth}
        \centering
        \includegraphics[width=\textwidth]{images/barbara_sous_echantillonage_sans_anti_aliassing.png}
        \caption{Sous-échantillonnage sans anti-aliasing} 
        \label{subfig:barbara-sans-anti-aliasing}
    \end{subfigure}
    \caption{Comparaison entre sous-échantillonnage avec anti-aliasing et
    sous-échantillonnage sans anti-aliasing de l'image \texttt{barbara.png}}
    \label{fig:filtrage-anti-aliasing}
\end{figure}

Lorsque la condition de Shannon n'est pas respectée, on est en situation de
sous-échantillonnage. Il peut alors se produire un repliement de spectre : il y
a perte d'information, on ne peut pas reconstruire l'image de départ et l'on se
retrouve avec un effet de crénelage.

Afin de pallier à ce problème, on effectue un filtrage anti-aliasing
préalablement au sous-échantillonnage. Les hautes fréquences, i.e. tous les
petits détails de l'image, seront bloquées (filtre passe-bas). Les zones de
haute fréquence seront donc lissées et paraîtront légèrement floues. Ce filtre
améliore donc la qualité de l'image.

\begin{figure}[H]
    \centering
    \begin{subfigure}[c]{0.46\textwidth}
        \centering
        \includegraphics[width=\textwidth]{images/mandrill_sous_echantillonage_avec_anti_aliassing.png}
        \caption{Sous-échantillonnage avec anti-aliasing} 
        \label{subfig:mandrill-avec-anti-aliasing}
    \end{subfigure}
    \begin{subfigure}[c]{0.46\textwidth}
        \centering
        \includegraphics[width=\textwidth]{images/mandrill_sous_echantillonage_sans_anti_aliassing.png}
        \caption{Sous-échantillonnage sans anti-aliasing} 
        \label{subfig:mandrill-sans-anti-aliasing}
    \end{subfigure}
    \caption{Comparaison entre sous-échantillonnage avec anti-aliasing et
    sous-échantillonnage sans anti-aliasing de l'image \texttt{mandrill.png}}
    \label{fig:filtrage-anti-aliasing}
\end{figure}

\section*{Exercice 4 - Éclatement d'une image couleur}

\begin{figure}[H]
    \centering
    \begin{subfigure}[c]{0.46\textwidth}
        \centering
        \includegraphics[width=\textwidth]{images/clown.png}
        \caption{Image \texttt{clown.bmp}} 
        \label{subfig:clown}
    \end{subfigure}
    \begin{subfigure}[c]{0.46\textwidth}
        \centering
        \includegraphics[width=\textwidth]{images/clown_lumi.png}
        \caption{Image \texttt{clown\_lumi.bmp}} 
        \label{subfig:clown_lumi}
    \end{subfigure}
    \caption{Visualisation des images \texttt{clown.bmp} et
    \texttt{clown\_lumi.bmp}}
    \label{fig:clowns}
\end{figure}

L'image \texttt{clown.bmp} (\figref{subfig:clown}) est en couleur (R,G,B) : il
est représenté par un tableau à 3 dimensions. L'image \texttt{clown\_lumi.bmp}
(\figref{subfig:clown_lumi}) est en noir et blanc.

La première image est donc trois fois plus grande que la seconde.

L’image I1 (\figref{subfig:clown}) possède les trois canaux R, G et B, qui
représentent l’image dans chacune des couleurs Rouge Vert et Bleu.

\begin{figure}[H]
    \centering
    \begin{subfigure}[c]{0.3\textwidth}
        \centering
        \includegraphics[width=\textwidth]{images/IR.png}
        \caption{Composante rouge} 
        \label{subfig:IR}
    \end{subfigure}
    \begin{subfigure}[c]{0.3\textwidth}
        \centering
        \includegraphics[width=\textwidth]{images/IV.png}
        \caption{Composante verte} 
        \label{subfig:IV}
    \end{subfigure}
    \begin{subfigure}[c]{0.3\textwidth}
        \centering
        \includegraphics[width=\textwidth]{images/IB.png}
        \caption{Composante bleue} 
        \label{subfig:IB}
    \end{subfigure}
    \caption{Composantes rouge, verte, et bleue de l'image \texttt{clown.bmp}}
    \label{fig:composantes}
\end{figure}

\begin{figure}[H]
	\center 
	\includegraphics[width=0.5\textwidth]{images/Question3RBG.png}
    \caption{Combinaison RBG}
    \label{fig:RBG}
\end{figure}

L'image contenue dans la \figref{fig:RBG} représente une combinaison RBG
(échange entre les plans vert et bleu). \\

On souhaite voir le plan rouge en rouge, le plan bleu en bleu et le plan vert en
vert. Les images obtenues, notées respectivement R, B et V sont contenues dans
la \figref{fig:plans}.

\begin{figure}[H]
    \centering
    \begin{subfigure}[c]{0.3\textwidth}
        \centering
        \includegraphics[width=\textwidth]{images/R.png}
        \caption{Plan rouge} 
        \label{subfig:IR}
    \end{subfigure}
    \begin{subfigure}[c]{0.3\textwidth}
        \centering
        \includegraphics[width=\textwidth]{images/G.png}
        \caption{Plan vert} 
        \label{subfig:IV}
    \end{subfigure}
    \begin{subfigure}[c]{0.3\textwidth}
        \centering
        \includegraphics[width=\textwidth]{images/B.png}
        \caption{Plan bleu} 
        \label{subfig:B}
    \end{subfigure}
    \caption{Plans rouge, verte, et bleu de l'image \texttt{clown.bmp}}
    \label{fig:plans}
\end{figure}

\end{document}
